% A LaTeX (non-official) template for ISAE projects reports
% Copyright (C) 2014 Damien Roque
% Version: 0.2
% Author: Damien Roque <damien.roque_AT_isae.fr>

\documentclass[a4paper,12pt,calibri,oneside,openany]{book}
\usepackage{geometry}
\usepackage[utf8]{inputenc}
\usepackage[T1]{fontenc}
%\usepackage[french]{babel} % If you write in French
\usepackage[english]{babel} % If you write in English
\usepackage{a4wide}
\usepackage{graphicx}
\graphicspath{{images/}}
\usepackage{subfig}
\usepackage{tikz}
\usetikzlibrary{shapes,arrows}
\usepackage{pgfplots}
\pgfplotsset{compat=newest}
\pgfplotsset{plot coordinates/math parser=false}
\newlength\figureheight
\newlength\figurewidth
\pgfkeys{/pgf/number format/.cd,
set decimal separator={,\!},
1000 sep={\,},
}
\usepackage{ifthen}
\usepackage{ifpdf}
\ifpdf
\usepackage[pdftex]{hyperref}
\else
\usepackage{hyperref}
\fi
\usepackage{color}
\hypersetup{%
colorlinks=true,
linkcolor=black,
citecolor=black,
urlcolor=black}
\usepackage{float}
\renewcommand{\baselinestretch}{1.05}
\usepackage{fancyhdr}
\pagestyle{fancy}
\fancyfoot{}
\fancyhead[LE,RO]{\bfseries\thepage}
\fancyhead[RE]{\bfseries\nouppercase{\leftmark}}
\fancyhead[LO]{\bfseries\nouppercase{\rightmark}}
\setlength{\headheight}{15pt}

\let\headruleORIG\headrule
\renewcommand{\headrule}{\color{black} \headruleORIG}
\renewcommand{\headrulewidth}{1.0pt}
\usepackage{colortbl}
\arrayrulecolor{black}

\fancypagestyle{plain}{
  \fancyhead{}
  \fancyfoot[C]{\thepage}
  \renewcommand{\headrulewidth}{0pt}
}

\makeatletter
\def\@textbottom{\vskip \z@ \@plus 1pt}
\let\@texttop\relax
\makeatother

\makeatletter
\def\cleardoublepage{\clearpage\if@twoside \ifodd\c@page\else%
  \hbox{}%
  \thispagestyle{empty}%
  \newpage%
  \if@twocolumn\hbox{}\newpage\fi\fi\fi}
\makeatother

\usepackage{amsthm}
\usepackage{amssymb,amsmath,bbm}
\usepackage{array}
\usepackage{bm}
\usepackage{multirow}
\usepackage[footnote]{acronym}
\usepackage{float}
\usepackage{wasysym}
\usepackage{wrapfig}
\usepackage{url}
\usepackage{eurosym}
\usepackage{array}
\usepackage{xcolor}
\usepackage{supertabular}
%\usepackage{geometry}
\usepackage{pdflscape}
\usepackage{calrsfs}
\usepackage{longtable, booktabs}

\newcommand*{\SET}[1]  {\ensuremath{\mathbf{#1}}}
\newcommand*{\VEC}[1]  {\ensuremath{\boldsymbol{#1}}}
\newcommand*{\FAM}[1]  {\ensuremath{\boldsymbol{#1}}}
\newcommand*{\MAT}[1]  {\ensuremath{\boldsymbol{#1}}}
\newcommand*{\OP}[1]  {\ensuremath{\mathrm{#1}}}
\newcommand*{\NORM}[1]  {\ensuremath{\left\|#1\right\|}}
\newcommand*{\DPR}[2]  {\ensuremath{\left \langle #1,#2 \right \rangle}}
\newcommand*{\calbf}[1]  {\ensuremath{\boldsymbol{\mathcal{#1}}}}
\newcommand*{\shift}[1]  {\ensuremath{\boldsymbol{#1}}}

\newcommand{\eqdef}{\stackrel{\mathrm{def}}{=}}
\newcommand{\argmax}{\operatornamewithlimits{argmax}}
\newcommand{\argmin}{\operatornamewithlimits{argmin}}
\newcommand{\ud}{\, \mathrm{d}}
\newcommand{\vect}{\text{Vect}}
\newcommand{\sinc}{\ensuremath{\mathrm{sinc}}}
\newcommand{\esp}{\ensuremath{\mathbb{E}}}
\newcommand{\hilbert}{\ensuremath{\mathcal{H}}}
\newcommand{\fourier}{\ensuremath{\mathcal{F}}}
\newcommand{\sgn}{\text{sgn}}
\newcommand{\intTT}{\int_{-T}^{T}}
\newcommand{\intT}{\int_{-\frac{T}{2}}^{\frac{T}{2}}}
\newcommand{\intinf}{\int_{-\infty}^{+\infty}}
\newcommand{\Sh}{\ensuremath{\boldsymbol{S}}}
\newcommand{\C}{\SET{C}}
\newcommand{\R}{\SET{R}}
\newcommand{\Z}{\SET{Z}}
\newcommand{\N}{\SET{N}}
\newcommand{\K}{\SET{K}}
\newcommand{\reel}{\mathcal{R}}
\newcommand{\imag}{\mathcal{I}}
\newcommand{\cmnr}{c_{m,n}^\reel}
\newcommand{\cmni}{c_{m,n}^\imag}
\newcommand{\cnr}{c_{n}^\reel}
\newcommand{\cni}{c_{n}^\imag}
\newcommand{\tproto}{g}
\newcommand{\rproto}{\check{g}}
\newcommand{\LR}{\mathcal{L}_2(\SET{R})}
\newcommand{\LZ}{\ell_2(\SET{Z})}
\newcommand{\LZI}[1]{\ell_2(\SET{#1})}
\newcommand{\LZZ}{\ell_2(\SET{Z}^2)}
\newcommand{\diag}{\operatorname{diag}}
\newcommand{\noise}{z}
\newcommand{\Noise}{Z}
\newcommand{\filtnoise}{\zeta}
\newcommand{\tp}{g}
\newcommand{\rp}{\check{g}}
\newcommand{\TP}{G}
\newcommand{\RP}{\check{G}}
\newcommand{\dmin}{d_{\mathrm{min}}}
\newcommand{\Dmin}{D_{\mathrm{min}}}
\newcommand{\Image}{\ensuremath{\text{Im}}}
\newcommand{\Span}{\ensuremath{\text{Span}}}

\newcommand{\anfr}[1]{{\bfseries\underline{#1}}}

\newtheoremstyle{break}
  {11pt}{11pt}%
  {\itshape}{}%
  {\bfseries}{}%
  {\newline}{}%
\theoremstyle{break}

%\theoremstyle{definition}
\newtheorem{definition}{Définition}[chapter]

%\theoremstyle{definition}
\newtheorem{theoreme}{Théorème}[chapter]

%\theoremstyle{remark}
\newtheorem{remarque}{Remarque}[chapter]

%\theoremstyle{plain}
\newtheorem{propriete}{Propriété}[chapter]
\newtheorem{exemple}{Exemple}[chapter]



%\sloppy
\usepackage{wrapfig}
\usepackage{enumitem}
\usepackage{pifont}
\usepackage{makeidx}
\usepackage{setspace}
\makeindex
\usepackage[xindy]{glossaries}
\makeglossaries
%\loadglsentries{glossaire.tex}




\begin{document}

\renewcommand{\bibname}{Bibliographie et Webographie}
%%%%%%%%%%%%%%%%%%
%%% First page %%%
%%%%%%%%%%%%%%%%%%

\begin{titlepage}
\begin{center}

\includegraphics[width=0.6\textwidth]{logohsb}\\[1cm]

%{\large Étudiants ingénieurs en aérospatial}\\[0.5cm]

%{\large DMSP}\\[0.5cm]

% Title
\rule{\linewidth}{0.5mm} \\[0.4cm]
{ \huge \bfseries Satellite Communication\\[0.4cm] }
\rule{\linewidth}{0.5mm} \\[1.cm]
\begin{center}
		Project 3 - POLAR Satellite
\end{center}
% Author and supervisor
\noindent
\begin{minipage}{0.4\textwidth}
  \begin{flushleft} \large
    \emph{Authors :}\\
    Emilio \textbf{\textit{Mitre-Perez}}\\
    Julien \textbf{\textit{Huynh}}\
  \end{flushleft}
\end{minipage}%
\begin{minipage}{0.4\textwidth}
  \begin{flushright} \large
    \emph{Supervising professor :} \\
    Prof. Soren \textit{Peik}\\
  \end{flushright}
\end{minipage}

\vfill

% Bottom of the page
{\large Version 0.1\\ \today}

\end{center}
\end{titlepage}

%%%%%%%%%%%%%%%%%%%%%%%%%%%%%
%%% Non-significant pages %%%
%%%%%%%%%%%%%%%%%%%%%%%%%%%%%

\frontmatter

%\chapter*{Remerciements}


\tableofcontents

\mainmatter
\pagestyle{fancy}
%%%%%%%%%%%%%%%%%%%%%%%%%%%%%%%%%%%%%%%%%%%%
%%% Content of the report and references %%%
%%%%%%%%%%%%%%%%%%%%%%%%%%%%%%%%%%%%%%%%%%%%



\chapter{General information}
\section{Polar Satellite}
The POLAR satellite is one of the 4 spacecraft launched for the GGS program (Global Geospace Science) which are part of the six spacecraft of the ISTP program (International Solar Terrestrial Physics). \\

\subsection{Mission and abilities}
POLAR is able to get multi-wavelength vision from the aurora, it measures the plasma entry to the polar magnetosphere as well as the geomagnetic tail, the flow both ways to the ionosphere and the displacement of energy particles into the ionosphere and the higher atmosphere. \\

\subsection{Orbit}
POLAR has a $22h$h and $36$ mins polar orbit with an apogee of $57\ 000$ km and a perigee of $11\ 500$ km. It was launched in 1996 to observe the polar magnetosphere and later was used to observe the equatorial inner

\subsection{Technical properties}
The POLAR satellite has a propulsion system and it is designed to have a lifetime of between 3 and 5 years and also has redundant subsystems. POLAR has a cylindrical shape with a $2.8$ m diameter base and $1.25$ m in height (plus $1.25$ m more for the despun platforms), it has solar cells to provide power, weights $1\ 250$ Kg and uses $333$ W of power. The spin rate of the satellite is $10$ RPM around and axis almost normal to the orbital plane. It also has long wire spin-plane antennas, spin-plane appendages to support the sensors and internal booms. The satellite has 2 despun gimbaled instrument platforms, and in the Z axes the booms are deployed.\\

The data is stored in tape recorders on-board and sometimes relayed to the Deep Space network at a maximum speed of 600 kbps and 41.6 kbps in average.\\
 magnetosphere.\\
\section{Ground Station}
Our ground station will be a small city called Bondy in France, our observer location is :
\begin{enumerate}
	\item Longitude : $2.478680$
	\item Latitude : $48.89976$
\end{enumerate}
\begin{center}
	\includegraphics[width=\linewidth]{bondy}
\end{center}
\chapter{Obtaining subpoints}
In order to obtain the subpoints of the satellite, we could either use the emphem module or the following formulas : \\
\underline{Latitude $\varphi$} :
$$
\varphi = \arcsin[\sin(i)\cdot \sin(\nu + \omega)]
$$
\underline{Longitude $\lambda$} :
$$
\lambda = \arctan[\tan(\omega + \nu)\cdot \cos(i)] - \bigg[\frac{\Omega_E}n (E-e\cdot\sin(E) - \frac{\Omega_E}{n}\cdot (E_N - e\cdot\sin(E_n))\bigg]
$$
As for the elevation and the azimuth relative to the observer, we can use the following equations  :
\underline{Elevation $E$} :
$$
 E = \arccos(\frac rR \sin(\phi))\\
 \cos(\phi) = \cos L \cdot \cos \varphi \cos l + \sin \varphi \sin l
$$
\underline{Azimuth $a$} :
$$
\sin(a) = \frac{\sin L \cos \varphi}{\sin \phi}
$$
For simplicity reasons, we will use the built-in functions for both the subplot and the computing of the elevation and azimuth with respect to the observer.
\chapter{Tracking the satellite}
\begin{center}
	\includegraphics[width=\linewidth]{groundtrack}\\
	\includegraphics[width=\linewidth]{azimuth}\\
	\includegraphics[width=\linewidth]{elevation}
\end{center}
\section{Link Budget}

	\subsection{Find the required data bit rate $R_{c}$ and symbol rate $R_{a}$ for transmitting the amount of 1000.0 MByte to the ground within one link window.}
	
		For the transmission, QPSK (quadrature phase shift keying) modulation is used. The modulation has 4 states so $M=4$ and $m$ can be calculated as $\log_2 M = 2$.
		
		For 16.25 hours of connection time, the required data bit rate $R_{c}$ and symbol rate $R_{a}$ can be obtained by the following expressions:
		
		\begin{equation} \label{eq_rc}
			R_{c}= \dfrac{\# \ of Bits}{T_{con}} = \dfrac{1000 \unit{MByte} \cdot 8 \cdot 10^{8}}{58500 \unit{s}}= 136752.14 \unit{bit/s} 
		\end{equation}
	
		\begin{equation} \label{eq_ra}
			R_{a}= \dfrac{R_{c}}{m} = \dfrac{136752.14 \unit{Bit/s}}{2}= 68376.07 \unit{symbol/s} 
		\end{equation}
	
	\subsection{Find the required bandwidth for the given modulation scheme and filtering.}	
	
		To calculate the bandwidth we will use the following expression:
		
		\begin{equation} \label{eq_ban}
			B= \dfrac{R_{c}}{\Gamma} 
		\end{equation}
	
		We need the value of the roll-off factor which is $\alpha=0.15$ to calculate the spectral efficiency $\Gamma$:
		
		\begin{equation} \label{eq_espectral_eff}
			\Gamma= \dfrac{m}{1+\alpha} = \dfrac{2}{1+0.15} =  1.74
		\end{equation}
		
		And therefore, going back to the equation \ref{eq_ban} the bandwidth value is:
		
		\begin{equation} \label{eq_ban2}
			B= \dfrac{136752.14}{1.74} = 7.86 \cdot 10^{-4} \unit{Hz}
		\end{equation}
	
	\subsection{Find the required energy per bit per noise $E_{c}=N_{o}$ for a BER of $10^{-6}$ and the corresponding SNR.}
	
		For the required BER of $10^{-6}$ and a QPSK modulation scheme, we will use the graph in Figure \ref{graf_BER} to obtain the Energy per bit per noise ratio $\left(\dfrac{E_{c}}{N_{o}}\right)$ and we get a value of $\dfrac{E_{c}}{N_{o}}\approx10.5 \unit{dB}$
	
		\begin{figure}[h]
			\includegraphics[width=14cm]{graph_BER}
			\caption{Bit Error Ratios for diferent Modulation Schemes Reference????}
			\label{graf_BER}
		\end{figure}
	
		Once we have the $\dfrac{E_{c}}{N_{o}}$ we can proceed to calculate the SNR:
		
		\begin{equation} \label{eq_snr}
			SNR = \dfrac{R_{c} \cdot \dfrac{E_{c}} {N_{o}}}{B} = \dfrac{136752.14 \cdot 10.5} {7.86 \cdot 10^{-4}} = 19,51
		\end{equation}
		
	\subsection{Find the total received noise at the antenna of the ground station $N_{i}$ and at the output of the receiver $N_{o}$.}	
		
		 To calculate the total received noise at the antenna of the ground station $N_{i}$ we can use the following expression: 
		 
		 \begin{equation} \label{eq_ni}
		 	N_{i} = k \cdot T_{a} \cdot B
		 \end{equation}
	
		In which the $k$ is the Bolzmann constant with a value of $ 1.38 \cdot 10^{-23} \unit{J/K}$, $B$ is the previously calculated bandwidth and $T_{a}$ is the antenna noise temperature, which can be found in the graph of Figure \ref{graph_Ta}.
		
		\begin{figure}[h]
			\includegraphics[width=14cm]{graph_Ta}
			\caption{Antenna noise temperature $T_{a}$ as a function of the Zenith angle and the frecuency. Reference????}
			\label{graf_Ta}
		\end{figure}
		
		
		To calculate the $T_{a}$ we will take the worst case scenario under consideration, which is for an elevation of $5^{\circ}$ (Zenith Angle $\Theta = 85^{\circ}$). In this case we get a $T_{a} \approx 30.5 \unit{K} $
	
		Now that we have calculated all the values we can put them back in Equation \ref{eq_ni}:
	
		\begin{equation} \label{eq_ni2}
			N_{i} = 1.38 \cdot 10^{-23} \cdot 30.5 \cdot 7.86 \cdot 10^{-4} = 4.21 \cdot 10^{-17} \unit{W}
		\end{equation}
	
		To obtain the input noise $N_{o}$ we will use the following calculation:
		
		\begin{equation} \label{eq_no}
			N_{o} = k \cdot \left(T_{a} + T{e} \right) \cdot B \cdot G_{r} 
		\end{equation}
	
		In which $k$ is the the Bolzmann constant, $T_{a}$ is the previously calculated antenna noise temperature, $T_{e}$ is the equivalent noise temperature, $B$ is the Bandwidth and $G_{r}$ is the gain of the receiver in the ground station. 
		
		To calculate the equivalent noise temperature $T_{e}$ we need the noise figure, which is given as $F=2 \unit{dB}$ and the initial temperature $T_{0}$ which is set at 290 K:
		
		\begin{equation} \label{eq_te}
			T_{e} = (F - 1) \cdot T_{0} = 169.62 \unit{K} 
		\end{equation}
	 	
	 	With all the parameters from Equation \ref{eq_no} known, besides the  gain of the receiver in the ground station which is set to a design value of $G_{r}= 100000$, we can calculate the $N_{o}$:
	 	
	 	\begin{equation} \label{eq_no2}
	 		N_{o} = 1.38 \cdot 10^{-23} \cdot \left(30.5 + 169.62 \right) \cdot 7.86 \cdot 10^{4} \cdot 100000 = 2.76 \cdot 10^{-11} \unit{W}
	 	\end{equation}
	 		
	\subsection{Find the required signal power $S_{o}$ for the calculated SNR and bandwidth. Find the required input signal power $S_{i}$ , i.e. the power received by the antenna.}
	
		To calculate the required signal power $S_{o}$ we use the following equation:
	
		\begin{equation} \label{eq_so}
			S_{o} = SNR \cdot N_{0} = 19,51 \cdot 2.76 \cdot 10^{-11} = 5,39 \cdot 10^{-10}  \unit{W}
		\end{equation}
	
		And to obtain the input signal power $S_{i}$ we will use the receiver gain that we assume before to be 100000 W (or 60 dB):
		
		\begin{equation} \label{eq_si}
			S_{i} = \dfrac{S_{0}}{G_{r}}= \dfrac{5,39 \cdot 10^{-10}}{100000} = 5,39 \cdot 10^{-15}  \unit{W}
		\end{equation}		
	
	\subsection{Find the required EIRP of the satellite for the worst case, i.e. satellite elevation is 5$^{\circ}$}
	
		To calculate the EIRP for the worst case scenario with an elevation of 5$^{\circ}$ we can use the following formula:
		
		\begin{equation} \label{eq_eirp}
		\begin{split}
			EIRP = SNR \cdot &L_{p} \cdot k \cdot B \cdot \dfrac{T_{a}+T_{e}}{G_{a}} = \\
			19.51  \cdot 3.84 \cdot 10^{20} \cdot 1.38 \cdot 10^{-23} \cdot &100000 \cdot \dfrac{30.5+169.61}{4.39 \cdot 10^{3}} = 451.31	 \unit{W}
		\end{split}
		\end{equation}
	
	\subsection{Design a transmitter with 30\% efficiency and a matching dish antenna. What is the	size of the antenna, how much DC power is required for the transmitter.}
	
		Finally we ought to design a transmitter with a 30\% efficiency and its antenna. For the transmitter we set a design value of 25 W. The first thing we have to calculate is the gain of the transmitter $G_{t}$:
		
		\begin{equation} \label{eq_gt}
			G_{t} = \dfrac{EIRP}{P_{t}}= \dfrac{451.31}{25} = 18.05
		\end{equation}
	
		Now that we have the gain of the transmitter, we will proceed to define the size of the antenna. For knowing its diameter $D$, we will have to know the phisical area of the antenna $A_{phy}$, and for that the effective area of the antenna $A_{eff}$:
		
		\begin{equation} \label{eq_aeff}
			A_{eff} = \dfrac{G_{t} \cdot \lambda^{2}} {4 \cdot \pi }
		\end{equation}
	
		Where lambda $\lambda$ is the wavelenght and can be calculated with the frequency $f$ and the speed of light $c$:
		
		\begin{equation} \label{eq_lambda}
			\lambda = \dfrac{c} {f}= \dfrac{2.998 \cdot 10^{8}} {8.35 \cdot 10^{9}} = 3.59 \cdot 10^{-2} \unit{m}
		\end{equation}
	
		Going back to the Equation \ref{eq_aeff}:
		
		\begin{equation} \label{eq_aeff2}
			A_{eff} = \dfrac{{18.05} \cdot (3.59 \cdot 10^{-2})^{2}} {4 \cdot \pi } =  1.86 \cdot 10^{-3} \unit{m^{2}}
		\end{equation}
	
		\begin{equation} \label{eq_aphy}
			A_{phy} = \dfrac{A_{eff}} {\eta_{t}} = 
			\dfrac{1.86 \cdot 10^{-3}} {0.3}=
			6.18 \cdot 10^{-3} \unit{m^{2}}
		\end{equation}
		
		\begin{equation} \label{eq_d}
			D = 2 \cdot \sqrt{\dfrac{A_{phy}}{\pi}} =
			2 \cdot \sqrt{\dfrac{6.18 \cdot 10^{-3}}{\pi}} = 8.87  \unit{m}
		\end{equation}
	
	\subsection{Create a table with the following specifications: frequency, Tx-Power, Tx-Antenna gain, Tx-EIRP, Tx dish diameter, Mod. Scheme, Bandwidth, Max. Distance to ground, Rx-Antenna gain, Rx-dish diameter, G/T of Rx, received power \\}	

		\begin{center}
			\begin{tabular}{ | l | l | l | p{1.5cm} |}
				\hline
				Description & Quantity & Value & Unit \\ \hline
				Frequency & f  & $8,35*10^{9}$ & Hz \\ \hline
				Tx-Power & Pt & 25 & W
				\\ \hline
				Tx-Antenna gain & Gt & $18.1$ & 
				\\ \hline
				Tx-EIRP & EIRP & $4.51*10^{2}$ & W
				\\ \hline
				Tx Dish Diameter & D & $8.87*10^{-2}$ & m
				\\ \hline
				Mod. Scheme & QPSK &  & 
				\\ \hline
				Bandwidth & B & $7.86*10^{4}$ & Hz
				\\ \hline
				Max. Distance to Ground & r & $5.60*10^{7}$ & m
				\\ \hline
				Rx-Antenna Gain & Gr & $10^{5}$ & 
				\\ \hline
				Rx-Dish Diameter & D & 1 & m
				\\ \hline
				G/T of Rx & G/T & 22.9 & $K^{-1}$
				\\ \hline
				Received Power & Prec & $5.39*10^{-15}$ & W
				\\ \hline
				
						\end{tabular}
					\end{center}
				
\newpage
\section{Power Budget}
	\subsection{Estimate the the required solar cell area and the battery capacity}
	
		To calculate the battery capacity we will need the power consumption $P_{tot}$ and the time that the satellite is in the shadow $t_{eclipse} = 3 hours$. For the power we have assumed that the platform requires 500 W when it is in the sun and 100 W when it is in the shadow.
		
		\begin{equation} \label{eq_cb}
			CB = \dfrac{P_{tot}}{t_{eclipse}} =
			\dfrac{P_{psun} +  P_{pshadow} + P_{t} }{t_{eclipse}}=
			\dfrac{500 +  100 + 25 } {3} = 208.33 \unit{Wh}
		\end{equation}
		
		For the required solar cell area $A_{sol}$ we will need the Irradiance $M_{sol}$ which has a value of $1367 \unit{W/m^{2}}$, the efficiency of the cell $\eta_{cell}$ which we set to 0.15 and the power required by the satellite when it is in the sun $P_{sol}$.
		
		To calculate the $P_{sol}$(Equation \ref{eq_psol}) we need to know first the power that it consumes to charge the battery $P_{charging}$, for which we need the period of the elipse $T_{elipse}$, the time on the shadow $t_{eclipse}$ and the efficiency of the battery $\eta_{bat}$, as well as the already calculated battery capacity $CB$.
		
		\begin{equation} \label{eq_pcharging}
			P_{charging} = \dfrac{CB}{\dfrac{T_{elipse}-t_{eclipse}} {\eta{bat}}	} =
			\dfrac{208.33}{\dfrac{18.5-3}{0.8}	} = 16.80 \unit{W}
		\end{equation}
		
		\begin{equation} \label{eq_psol}
			P_{sol} = P_{charging} + P_{psun} + P_{t}= 
			16.80 + 500 + 25= 541.80 \unit{W}
		\end{equation}
		
		Now that we have all the values we can calculate the required solar cell area $A_{sol}$:
		
		\begin{equation} \label{eq_asol}
			A_{sol} = \dfrac{P_{sol}}{M_{sol} \cdot \eta_{cell}}= \dfrac{541.80}{1367 \cdot 0.15}=  2.64 \unit{m^{2}}
		\end{equation}
		
	
	





%\glsaddall
%\printglossaries

%\nocite{*}

	
%	\bibliographystyle{plain} % Le style est mis entre accolades.
%	\bibliography{references} % mon fichier de base de données s'appelle bibli.bib

%\printbibliography



%\include{lexique_an_fr}
%\listoffigures

%\listoftables
%\nopagebreak
%\include{annexes}

%\printindex

\end{document}
